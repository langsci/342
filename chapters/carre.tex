\documentclass[output=paper,colorlinks,citecolor=brown,
% hidelinks,
% showindex
]{langscibook}

\author{Alice Carré\affiliation{Université Grenoble-Alpes} and Dorothy Kenny\affiliation{Dublin City University} and Caroline Rossi\affiliation{Université Grenoble-Alpes} and  Pilar Sánchez-Gijón\affiliation{Universitat Autònoma de Barcelona} and Olga Torres-Hostench\affiliation{Universitat Autònoma de Barcelona}}

\title{Machine translation for language learners}
\abstract{Machine Translation (MT) has been controversial in second and foreign language learning, but the strategic integration of MT might be beneficial to language learning in certain contexts. In this chapter we discuss the conditions in which MT can be useful in language learning, set out digital alternatives to MT, and provide examples of how MT can support language learners.}

\ChapterDOI{10.5281/zenodo.6760024}

\begin{document}
\lehead{A. Carré, D. Kenny, C. Rossi, P. Sánchez Gijón \& O. Torres-Hostench}
\AffiliationsWithIndexing{}
\maketitle

\section{Introduction}\label{sec:carre:1}

Machine translation (MT) has been controversial in second and foreign language learning,\footnote{Note that we use the generic terms \textit{language learning} and \textit{language learner} in this chapter to cover instances of foreign language learning and second and subsequent language learning. If a student’s first language is their L1, then the language learning to which we refer corresponds to their learning of an L2, L3 or L\textit{n}.}  with some commentators arguing that it can encourage plagiarism, promote errors or deflect learners from what they should be doing. In some cases, however, MT has been found to help students complete certain tasks, and there appears to be merit in considering MT as just one among many digital resources that contemporary language learners can use. The successful integration of MT into language learning requires us to understand, even at a basic level: how the technology works, how we can judge the quality of its outputs, how those outputs can be improved through intervention either before or after the fact of translation (through pre-editing or post-editing), and what the ethical issues in using MT are, among other factors. These factors, which are often subsumed under the heading of \textit{MT literacy} \citep{BowkerCiro2019} have been covered in depth in Chapters 2 to 6 of this book. \textcitetv{chapters/torres}, meanwhile, presented compelling arguments as to why MT needs to be considered as a vital building block of multilingual societies \textit{alongside} language learning. What has been missing so far is a deeper engagement with the use of MT \textit{within} language learning. In this chapter we aim to complete the jigsaw by addressing precisely this issue. We start by looking at the role of translation in language learning, and then ask whether it is acceptable to use MT for this purpose, and what benefits can be gained by doing so. We go on to suggest contexts in which language learners should or should not use MT, depending on a number of contextual parameters, and given the other, often more appropriate digital resources available to them. Finally, we give practical examples of how MT can be used in language learning contexts. 

\section{Translation in language learning}\label{sec:carre:2}

Before considering the role of MT in language learning, it is worth remembering that the role translation itself might play in language learning has long been a matter of debate. The much criticized grammar-translation method, in which learners were asked to learn vocabulary and grammar rules before translating sentences out of context, is generally rejected as a narrow pedagogical use of translation, but this rejection is itself based on a very narrow understanding of translation. At the same time, there is renewed interest in what \citet{Cook2010} aptly dubbed TILT (Translation in Language Teaching) and “studies exploring translation in the language classroom have multiplied in the last decade within a variety of disciplines” \citep[12]{PintadoGutiérrez2018}. Reported benefits include improved “plurilingual, pluricultural and communicative competences” as well as better writing skills, language awareness and control (ibid.: 13). The use of translation in the language classroom has also been shown to reduce anxiety and cognitive load among language learners (\citealt{KellyBruen2017}). The vibrancy of TILT is exemplified in sources such as \citet{CarreresGutiérrez2021}, which presents the latest trends in the integration of translation-related activities in the language curriculum.

\section{Machine translation in language learning}\label{sec:carre:3}
\subsection{Is it acceptable to use MT in language learning?}

Research conducted in the 2010s suggests that MT was, by and large, still taboo in the language classroom at the time. Based on the results of a survey of teachers worldwide, \citet{PymMalmkjaer2013} showed that very few reported working with machine translated texts in their language classes. There also appeared to be a causal link between MT and reluctance to use translation in language teaching, as reported by one informant in the study: in secondary schools the decision not to use translation was often based on “fear of students preferring to rely on machine translation tools” \citep[93]{PymMalmkjaer2013}. 

\begin{sloppypar}
There is, however, a growing realization that the use of Free Online MT (FOMT) is widespread among language learners, regardless of what teachers think, and that the effects of this use are worthy of investigation. Researchers such as \citet{Lee2021} and \citet{JolleyMaimone2022} review the burgeoning literature on the use of MT in language learning. Among the trends they recognize is the tendency to view the use of MT as cheating. This has led some commentators to write about how MT use could be “detected” in students’ L2 writing. When MT quality was generally poor, its use was easily detected in the kind of mistakes that were typical of the technology. Nowadays, researchers now argue that:\end{sloppypar}

\begin{quote}
As MT technologies continue to improve, identifying translation “mistakes” will likely become increasingly difficult for language instructors. Instead, it will be the technology's subtle successes, rather than its breakdowns, that will signal MT use. (\citealt{DucarSchocket2018}: 787)
\end{quote}

In other words, students will get caught out not because their writing is riddled with errors, but because it is too good for their level. A beginner learner of French who produces a subjunctive verb form that would not normally be encountered until they had reached an advanced stage, for example, might thus be suspected of using MT.

But whether someone is cheating or not depends not on the technology they are using, but on the rules of the game. If learners are forbidden from using MT in their L2 writing, but nonetheless use the technology surreptitiously, then that is cheating. Even if they are not expressly forbidden from using MT, but use it without letting the teacher know, and with the intention of passing the MT output off as their own writing, then this is still a dishonest action that is carried out to gain some kind of advantage. Indeed, the presentation of “someone else’s words” as one’s own belongs to the category of cheating known as \textit{plagiarism}, a topic that is addressed by \citet{MundtGroves2016} in the context of MT use in language learning. A number of studies (e.g. \citealt{Correa2011, Clifford2013, DucarSchocket2018}) show, however, that attitudes to the use of MT in language learning can differ between learners and teachers, or depending on the extent of the use of MT, or a host of other issues, and so the situation may not be as clear cut as it first seems. 

If you are learning a language in a formal setting, the best advice is to talk to your teacher, to make sure you understand what does and does not constitute cheating in your particular circumstances. If you are a language teacher, the best advice is to talk to your students to ensure that they know what is expected of them. Either way, the wish to avoid cheating is just one thing that needs to be taken into consideration when deciding whether or not it is acceptable to use MT in a language learning assignment. Other considerations are listed under “situational parameters” in \sectref{sec:carre:3} of this chapter, and in \textcitetv{chapters/moorkens} on ethics. For now, we limit ourselves to a discussion of the nature of the “advantage” that might be gained in using MT, with or without the approval of a teacher.

\subsection{What do you gain by using MT in language learning?}

There is some evidence that learners can gain an advantage in the short term by using MT to accomplish particular tasks. In the study reported on by \citet{O’Neill2019}, for example, 310 American intermediate-level university students wrote short compositions in French and Spanish, under different conditions, involving the use of: Google Translate with prior training for students; Google Translate without prior student training; an online dictionary with prior training; an online dictionary without prior training; or no technical aid at all. The students who used Google Translate, and had prior training in the use of the tool, scored better than all others on their compositions, followed by those who used an online dictionary, again with prior training. In a post-test conducted about a week later, and a delayed post-test conducted some three to four weeks later, where students no longer had access to the tool in question, the Google Translate + Training group no longer performed better than the other groups. It appears that any advantage they had gained by using the tool was short-lived and dependent on the continued availability of the tool. 

In a separate study, \citet{Fredholm2019} tracked lexical diversity in compositions written, over the course of a full school year, by 31 Swedish upper secondary school pupils of Spanish as a foreign language, in which roughly half the pupils used a printed dictionary as a translation tool, and the other half used Google Translate. He found that use of MT was associated with higher lexical diversity, and hence better performance, as long as students continued to have access to the tool, but once access was removed, the effect vanished. Again, the benefit bestowed by use of MT seemed dependent on the continued availability of the tool.

So does this mean that it is not worthwhile using MT in language learning? Not quite. In both studies, the use of MT did not harm students in the long run; there was simply no difference between students who used MT and students who did not, once the tool was no longer accessible. In the short term, however, the students who used MT did better than the others. So whether you benefit or not from MT use appears to depend on whether you take a short or long term view, and whether you focus on a particular task as an end it itself or on your development as a language learner.

Another lesson from \citegen{O’Neill2019} study is that – in the short term at least – training matters. Learners who are trained, even briefly, in how MT works, write better compositions than those with no training.

\subsection{What, exactly, can MT help you with?}

We have already seen that the use of MT has been shown to help certain learners write better compositions generally, or write compositions with more diverse vocabulary than other learners. In general, studies that track the effect of MT use on L2 (or L3) written composition tend to focus on learners’ use of \textit{vocabulary}, \textit{grammar}, and \textit{syntax}. According to \citet{Lee2021}:

\begin{quote}
Numerous studies have confirmed that MT helps students reduce orthographic, lexical, and grammatical errors and focus more on content, and as a result, students with MT produce revisions with a greater number of successful edits and a better quality of L2 writing. \citep[4]{Lee2021}
\end{quote}

But it should be noted that individual studies can produce seemingly conflicting results. \citet{Fredholm2015}, for example, found that Swedish pupils who used FOMT in their written compositions in Spanish made fewer mistakes in spelling and article/noun/adjective agreement, but more mistakes in syntax and in verb conjugation, than pupils who did not use FOMT. 

Other studies are interested in learners’ \textit{metalinguistic awareness}, defined as:

\begin{quote}
 {the ability to focus attention on language as an object in and of itself, to reflect upon language, and to evaluate it. (\citet[531]{Thomas1988} in \citealt[67]{ThueVold2018})}
\end{quote}

\citet{EnkinMejías-Bikandi2016} propose (but do not test) exercises in which students are presented with machine translations into Spanish of sentences involving “structures of interest” in English, and where contrastive differences mean that MT traditionally has not been very successful. This is the case, for example, with non-finite subordinate clauses that are best translated into Spanish using finite subordinate clauses. The idea is that students can reflect on where the machine goes wrong, thus honing their own metalinguistic, and especially contrastive awareness. The authors note, however, that as MT improves, “materials may need to be updated” \citep[145]{EnkinMejías-Bikandi2016}. It is probably fair to say, however, that neural MT engines for language pairs like English-Spanish have already reached such a level of quality that it is no longer reasonable to expect them to translate any given structure of interest incorrectly, as a matter of course, and that the kind of exercise envisaged by these authors needs to be rethought, so that students are encouraged to reflect on the successes, rather than the failures, of MT.\largerpage

The same observation can be made about studies that rely on learners correcting errors in MT output. Not only can exposure to “bad models” \citep{Niño2009} be controversial in language learning, but it might also be increasingly difficult to spot errors in contemporary neural MT in the first place (\citealt{CastilhoWay2017, LoockLéchauguette2021}), making post-editing type tasks (see \citetv{chapters/obrien}) less suitable for use in certain language learning contexts than was previously the case (cf. \citealt{ZhangTorres-Hostench2019}).\footnote{Having said that, recent studies, like that conducted by \citet{LoockLéchauguette2021}, may be more interested in developing MT literacy -- rather than metalinguistic awareness per se -- among language learners, and \textit{teacher-guided} error analysis of MT output may serve this purpose well.} 

\citet{ThueVold2018} reports on another study on metalinguistic awareness. This time learners of French as an L3 in an upper secondary school in Norway had to read two different machine-translated versions of the same text (one translated by Google Translate, the other by Microsoft’s Bing Translator), decide which machine translated version was better and explain why. The exact proficiency level of the students was not ascertained, although the author \citep[73]{ThueVold2018} intimates that it was unlikely to be above B1 on the Common European Framework of Reference for Languages.\footnote{\url{https://www.coe.int/en/web/common-european-framework-reference-languages}} Thue Vold concludes that while the use of MT texts to develop learners’ metalinguistic awareness has “considerable potential”, “training, scaffolding techniques and guidance from the teacher are of paramount importance” (ibid.: 89) as, left to their own devices, learners may not explore fruitful avenues of analysis, and their group conversation may even reinforce misconceptions about language (ibid.).

\subsection{Tips on using MT in language learning}\largerpage

The jury is still out on the precise benefits of using MT in language learning. It is likely, however, that such benefits depend on a whole host of factors including students’ proficiency levels and the text genres used in L2 writing tasks (see \citealt{ChungAhn2021}), as well as the language pair concerned (see \sectref{sec:carre:4.1} below). What is increasingly agreed upon in existing research is, however, that:

\begin{itemize}
\item Language learners use MT and rather than trying to outlaw its use, it is better to take a nuanced approach, based on an understanding of where MT can be more or less helpful, depending, perhaps, on the extent and context of use.
\item Language learners make better use of MT when they have received appropriate training.
\item Language learners can generally benefit more from MT if they already have reasonably good proficiency in the foreign language (\citealt{O’Neill2012, ResendeWay2021}).
\end{itemize}

Assuming that we do embrace MT in language-learning contexts there are some basic points that should be noted by both teachers and learners, as they may not be self-evident (see also \citealt{Bowker2020}).

First, there is more than one option available to learners who wish to use MT. Many learners appear to be aware only of Google Translate (see, for example, \citealt{DorstBouman2022}), but there is much to be learned by comparing the outputs of different FOMT systems, such as Bing Translator, DeepL or Baidu.

Second, MT outputs change over time as engines are re-trained or improved by users. This means that systems should not be written off based on a single use. It also means that researchers who use FOMT in their publications should always say exactly when they created the outputs in question, but this is rarely done.

Third, if you are using a FOMT system to create a language-learning exercise, to make a point about the technology, or to help you with a written composition, you should make sure that you give the system a fair chance to succeed. Very slight changes in input can have a big impact on outputs. An example may help here: in a generally very helpful discussion of MT in L2 learning,  \citet[785]{DucarSchocket2018} present an example of where Google Translate apparently produces a poor output based on “the more frequent and literal meaning of the word milk rather than the intended metaphorical meaning.”  The example is reproduced here in \figref{fig:carre:1}.

  
\begin{figure}
\includegraphics[width=\textwidth]{figures/MultiTraiNMTChapter9MTforLanguageLearners-img001.png}
\caption{\label{fig:carre:1}: input to Google Translate with sentence-initial lower case and no sentence-ending punctuation, produced 19 October 2021}
\end{figure}


What is notable here is that the input does not have a capital letter at the beginning of the sentence and there is no full stop at the end. If these features of the standard written language are reinstated, however, the output also changes – for the better – as shown in \figref{fig:carre:2}.


  
\begin{figure}
\includegraphics[width=\textwidth]{figures/MultiTraiNMTChapter9MTforLanguageLearners-img002.png}
 \caption{input to Google Translate with sentence-initial upper case and sentence-ending punctuation, produced 19 October2021}
\label{fig:carre:2}
\end{figure}

Similar issues have been observed when students copy-paste text into a FOMT window, not realizing that they may have done so in such a way that each line has a line break at the end of it, and what the FOMT thus sees is a series of independent lines, each of which will be translated independently of the others. State-of-the-art MT engines are trained to translate sentences. They work best when they can actually identify and translate full sentences. It is therefore important to make sure that you don’t “feed” text full of stray line breaks to the machine.

\begin{itemize}
\item Just as the outputs of different MT engines or systems can be fruitfully compared with each other, the usefulness of MT in language learning can be fruitfully compared with the usefulness of competing or complementary tools, such as corpus tools or online dictionaries. The next section elaborates on this point.
\end{itemize}

\section{When to use MT: linguistic and situational parameters}\label{sec:carre:4}
\subsection{I get by with a little help} \label{sec:carre:4.1}

So, if you are learning a foreign language, or dealing with a foreign language that you don’t know very well, is MT your best friend? It will appear to be an easy and quick solution, and one that could even help you trick your teacher or addressee into believing that your mastery of the language is quite good, so much so that it is not unusual these days for modern language students to share love stories about free MT engines, acknowledging for instance that the system made them “bilingual for an hour”.\footnote{This is just one of our students’ MT stories: \url{https://mtt.hypotheses.org/our-students-mt-stories}}  Can you, however, use MT to improve your foreign language writing, reading, listening and conversation skills? As indicated above, research shows that you need two things for this improvement to happen: first, reasonably good proficiency in the foreign language, and second, sound knowledge of MT and a set of skills now often described as “machine-translation literacy” \citep{BowkerCiro2019}, as indicated above. While the former can only be achieved through repeated practice, the principles and advice presented in this section will show you how to develop the latter.

\subsection{Language pairs and genres} 

To begin with, you need to consider that MT may be very good with some language pairs, and less good with others. Indeed, because NMT systems are corpus-based (as explained in Chapters 2 and 7), they will produce poorer results when too little data is available to train the system. In the examples given in this Section, we use the FR<>EN language pair (looking at translations from French into English as well as from English into French), for which current MT solutions often produce good enough results, but we certainly encourage readers to find examples in their language pairs and compare them with ours. 

Genre also makes a difference. You may find for instance that FOMT is better at translating essays than it is at translating poems or the lyrics of your favourite song. This may be because the data used to train the MT system are more similar to the former, and translated songs and poetry are probably quite rare in the training data. Poem and song translation are also particularly demanding: translated poems and songs may have to be recitable or singable. They may require particular rhyming schemes or metres. Although machines can be trained to write and even to translate poetry \citep{VandeCruys2018, VandeCruys2019, VandeCruys2020}, general-purpose FOMT might not be up to the task. It is still an interesting exercise to try it out, however: take a popular song, poem or nursery rhyme in either your L1 or L2. Find a good human translation of it,\footnote{You may be able to find published human translations of poems, songs and nursery rhymes online.} one that tries to create pleasing rhymes and rhythm in the target language. Now run the original through a FOMT engine and compare the MT with the human translation. The results are likely to encourage you to reflect on what MT does well, and what human translators do wonderfully.

\subsection{Ask a linguist}

Second, you need to consider your expectations and those of your teacher and/or interlocutor. It’s always a good idea to ask them whether MT is an acceptable solution, and whether it might interfere with your learning in ways that could be inappropriate. There might be cases in which the material used by your teacher cannot be put into a FOMT engine, because it contains personal or confidential data (as explained in \citetv{chapters/moorkens}). A language teacher could also tell you that MT will prevent you from learning grammar rules as it partly deprives you of the possibility of actively finding the right structure and phrasing. And if proficient use of grammar is needed to fix MT outputs, you won’t necessarily become more proficient as you interact with MT. Quite the opposite: exposure to mistakes or approximations that might go unnoticed by learners could be detrimental. Learning about a foreign language based on repeated exposure to fluent MT outputs might occur, however, in at least some cases: in a recent study, \citet{ResendeWay2021} evidenced implicit learning about syntax from NMT outputs in some of the participants. It remains to be seen whether learners will also be influenced by the errors in the MT. Without good MT literacy and good-enough knowledge of the target language, it is more likely that they simply won’t see the mistakes. (See, for example, \citealt{LoockLéchauguette2021}.)

Our advice, therefore, is that you should always ask a linguist: your teacher is a good start, as they will have a good idea of your L2 language proficiency and of the good and bad sides of using FOMT for a given language pair in a given context.

\subsection{Situational parameters}

In order to help language learners to decide when to use MT and when not, \tabref{tab:carre:1} includes a short list of situations with suggested decision parameters. Our proposal is based on the use of free online NMT solutions. (Note: If the answer to parameter 1 in \tabref{tab:carre:1} is negative, then don’t use MT.)


\begin{table}
\begin{tabularx}{\textwidth}{QQ}
\lsptoprule
{Situation} & {Decision parameters, ranked by order of importance}\\
\midrule
Understanding an L2 text & 1. No personal or confidential data in the text

2. Quality of the NMT output (which does not have to be perfect, but should be good enough)\\
\tablevspace

Writing an essay in your L2 & 1. Teacher’s agreement

2. No personal or confidential data in your essay

3. Sufficient L2 proficiency (B1 or B2 in CEFR)

4. Quality of the NMT output\\
\tablevspace
Performing a translation assignment & 1. Teacher’s agreement (unlikely because using MT transforms the assignment from a translation task to a post-editing task)

2. No personal or confidential data in your source text

3. Quality of the NMT output\\
\tablevspace

Getting ready for an oral presentation in your L2 & 1. Teacher’s agreement

2. No personal or confidential data in your presentation

3. Sufficient L2 proficiency (B1 or B2 in CEFR)

4. Quality of the NMT output and availability of good text-to-speech output\\
\lspbottomrule
\end{tabularx}
\caption{When to use MT and when not}
\label{tab:carre:1}
\end{table}

\section{Machine translation and competing digital resources}

FOMT is quick and easy to use, and because you can use full texts as inputs, you might be content with what you get and tempted to look no further. But more often than not, MT won’t be enough, and in some situations it might even be inappropriate. How does MT compare to other digital resources? Here is a list of four questions that will be discussed in turn in what follows, with a view to shedding light on what MT is, and what it is not. Gathering answers to these questions is a good way of starting a discussion on uses of MT versus other online tools: 

\begin{itemize}
\item Do you use online dictionaries, and if so, which ones?
\item How would you define a corpus?
\item Have you used an online corpus before?
\item Do you know what a concordance is?
\end{itemize}

The aim of the series of comparisons presented in what follows is to help L2 learners and foreign-language users in general reach beyond the immediacy of MT. One of the key points that we would like to make with these comparisons is the following: just because they are easy to use and provide you with almost instant translations does not mean that NMT solutions should be the preferred choice all the time.

\subsection{MT versus online dictionaries}

Generally speaking, a dictionary is “a book that contains a list of words in alphabetical order and explains their meanings, or gives a word for them in another language” (\citetitle{Cambridge2020}, \citeyear{Cambridge2020}). The definition further extends to electronic products like online dictionaries or apps, which are usually well-known to language learners. Examples of online dictionaries include more traditional ones like the Oxford English Dictionary,\footnote{\url{https://www.oed.com/}}  and new forms in which at least part of the information has been crowdsourced, such as Wiktionary\footnote{\url{https://en.wiktionary.org/}, last accessed 20 June 2022.} or Urban Dictionary.\footnote{\url{https://www.urbandictionary.com/}, last accessed 20 June 2022.}

Whether you have already been using them on a regular basis or not, the question for every user faced with such varied resources is: how can you tell that a dictionary is reliable? Lexicography (the writing of dictionaries) defines best practices and is constantly evolving to take into account the development of new dictionary forms and formats. Many dictionaries are now based on large collections of texts (also known as corpora) that can be used as references, but a lexicographer will never be happy with a mere quote from a corpus. Instead, corpora are often used to check language use and find relevant examples. Before doing so, a lexicographer’s job crucially involves finding the relevant entries for a given dictionary, and writing up good definitions. To help you appreciate what this involves, learners’ dictionaries are particularly telling because they include a really careful selection of entries, definitions and examples. With most dictionaries now freely available online, it should be easy for you to identify one or two learners’ dictionaries and look for words that you have learnt recently or that you particularly like. 

Here is another important point: the Linguee website claims to include a dictionary, but what kind of dictionary is it? Distinguishing between corpus-based and corpus-driven dictionaries may help: while corpus-based dictionaries still rely on a lexicographer’s intuitions and are built according to the methods of lexicography, corpus-driven dictionaries are based on automatic extractions from a corpus. Linguee includes a corpus-driven, bilingual dictionary that does not provide you with definitions of words or carefully selected examples. 

Before looking at online corpora, let us sum up the main differences between an NMT output and a dictionary entry:\largerpage[-2]

\begin{itemize}
\item  Dictionary entries are based on a single word, while you can get an NMT output for as much text as you like. NMT engines are far less useful than dictionaries because their output for an isolated word is often unreliable.\footnote{We say this in the knowledge that language learners and university students in general do, in fact, use FOMT engines very frequently to find translations of single words (see, for example, \citealt{JolleyMaimone2022, DorstBouman2022}).}
\item  Dictionaries provide you with definitions, which may be the only reliable way to make sure you have understood the meaning of a word.
\item  Dictionary entries are based on human intuition and (most of the time) they are designed and/or checked by lexicographers.
\item  NMT outputs are based on corpora, but they are not exact quotes from the corpora that have been used for training (as explained in \citetv{chapters/perez}). The next section addresses this difference in more detail.
\end{itemize}

\subsection{MT vs online corpora} 

At least two definitions are needed before we start explaining the differences between MT and online corpora. 

First, a parallel corpus is a collection of source texts aligned with their translations. This parallelism or alignment means that each segment (usually a sentence) appears next to its translation. 

A concordancer is the tool that is used to look for data in corpora and to display results. \tabref{tab:carre:2} contains an example from the Hansard corpus, a parallel corpus (English and French) of debates in the House and Senate of the Canadian Parliament.


Note first that the concordances are displayed as a function of the search: here we have looked for a French phrase, with a translation into English, so French appears first (that is, on the left), even though the source text for this part of the corpus is English (information that is not always displayed when using online corpora). It is also worth saying that bilingual concordancers usually highlight the exact search query (in bold in \tabref{tab:carre:2}) in the source segment. Many also attempt to highlight the part of the target segment that corresponds to the search query, but the identification of such correspondences is based on a kind of probabilistic “guesswork”, and is often not completely reliable. We have, in fact, “cleaned up” the target side of \tabref{tab:carre:2}, to make the “equivalent” (see \citetv[§1]{chapters/kenny}) parts of the concordance lines clearer. In short, alignment might be accurate at sentence level, but fine-grained correspondences are not always identified automatically in bilingual or parallel concordancers.

Note also that online concordancers often display the full paragraph or text for each concordance, as shown in \tabref{tab:carre:2}, or links to the relevant part of the corpus.

In contrast, an MT output will provide you with a proposed translation only for the words in your search, with no other contextual elements. 


\begin{table}
\begin{tabularx}{\textwidth}{QQ}
\lsptoprule
{French translation} & {English source text}\\
\midrule
Nous vous disons que \textit{nous allons faire le nécessaire}, mais aidez-nous à nous assurer que tout le monde respecte les règles.

Monsieur le Président, nous avons promis que \textit{nous allions faire le nécessaire} pour ratifier l'accord. & But we're saying, hey, \textit{we'll do it, we'll set it up}, but help us to make sure everybody abides by the rules.

Mr. Speaker, we promised that \textit{we were going to do what was required} to ratify the agreement.\\
\lspbottomrule
\end{tabularx}
\caption{Sample concordance lines for \textit{nous allons faire le nécessaire} in the Hansard parallel corpus\label{tab:carre:2}}
\end{table}

\begin{table}
\begin{tabularx}{\textwidth}{QQ}
\lsptoprule
{French query} & {English NMT output}\\
\midrule
Nous allons faire le nécessaire.

Nous allons faire le nécessaire pour ratifier l’accord. & We will do what is necessary.

We will take the necessary steps to ratify the agreement.\\
\lspbottomrule
\end{tabularx}
\caption{Sample NMT output for \textit{nous allons faire le nécessaire} (MT by \url{https://www.bing.com/translator}, 2021-11-01)}
\label{tab:carre:3}
\end{table}

\tabref{tab:carre:3} presents a comparison of two queries, the first one being shorter and more ambiguous than the second. It shows that NMT engines are able to adjust to sentence contexts: linked with the explicit mention of an objective (\textit{pour ratifier l’accord}) a different construction is used in the English NMT output, where \textit{faire le nécessaire} becomes \textit{take the necessary steps to}.

Overall, it makes much more sense to use NMT with full sentences (see \citetv[§7]{chapters/kenny}) or texts than with isolated words or phrases. When looking for a word, a collocation or a phrase, it might be more efficient and reliable to use a dictionary and/or a corpus, since you will get controlled results. Parallel corpora give you access to a series of translation choices whose context is usually easy to retrieve. MT, on the other hand, outputs results that are based on complex computations from training data that are not always accessible (they are typically hidden in FOMT interfaces). This can make it difficult for users to determine whether a proposed translation is indeed reliable.

\section{Error analysis}

One of the key skills that is needed to leverage MT for the purposes of second or foreign language learning is a keen awareness of errors. Various activities might be set forth in order to develop this skill, but because proposals are still scarce, we provide readers with one commented example in what follows.\footnote{Further recent ideas on the integration of MT into language teaching and learning can be found in \citet{VinallHellmich2022}.}

Based on a text and its MT, learners make a list of the types of errors they are able to detect and correct in the language pair in question. They submit the list to their teacher and then receive their teacher’s feedback and a second list containing all the errors they hadn’t noticed, with additional explanations, suggestions for further improvements and helpful examples. This exercise will be difficult if the target (machine translated) text is in the L2, and we suggest that teachers might start with translation into the mother tongue. While such tasks have been excluded from the language classroom for a long time, as indicated in the earlier part of this chapter, recent proposals integrating translation into situated tasks have been made, with a view to turning the learner into a “self-reflective, interculturally competent and responsible meaning maker in our increasingly multilingual world” \citep[105]{Laviosa2014}. 

\tabref{tab:carre:4} contains an example of an NMT output for a short text translated from English into French.\footnote{The text is taken from a textbook for French learners of English (\citealt{Joyeux2019}:22).  In order to turn this activity into a situated task, learners could be asked to provide a good translation to a French person with virtually no knowledge of English (e.g. a visitor to the class on a special occasion). They would need to receive minimal information about MT and about the need to correct the output that has been provided to them.}  Errors in the NMT output are highlighted in bold and commented on below.\footnote{Error analyses of MT output generally depend on \textit{error typologies}, which list various types of problems that can be found in MT output. These usually incorporate accuracy errors (e.g. the meaning of the target segment is not consistent with that of the source segment) and errors that affect the fluency or well-formedness of the target segment (e.g. errors in grammatical agreement, word order, collocation, etc.). For more information, see \textcitetv{chapters/rossi} on MT evaluation and \textcitetv{chapters/obrien} on post-editing.}


\begin{table}
\begin{tabularx}{\textwidth}{L{5.1cm}L{6.5cm}}

\lsptoprule
{English source text} & {French NMT output}\\
\midrule
Hi there!

My teacher asked me to write and tell you what a typical day in my life looked like so I’ll do my best to give you an idea! I get up around 7:00 am, I have breakfast (two slices of toast and a cup of tea) then I get ready and put on my uniform. I make my lunch, and I double-check that my bag is packed. I leave my house at around 8:00 am. I’ve only got a 15-minute walk to school, so I arrive early. I usually chat with my friends, or listen to music with my headphones. Classes begin at 8:45 am. & Bonjour à tous !

Mon professeur m'a demandé de vous écrire et de vous raconter à quoi ressemblait \textit{une journée typique de ma vie}, alors je vais faire de mon mieux \textit{pour vous donner une idée} ! Je me lève vers 7 heures du matin, je prends mon petit déjeuner (deux tranches de pain grillé et une tasse de thé) puis je me prépare et je mets mon uniforme. Je prépare mon déjeuner, et je vérifie que mon sac \textit{est bien emballé.} Je quitte \textit{ma} maison vers 8 heures. \textit{Je n'ai que 15 minutes de marche pour me rendre à l'école, donc j'arrive tôt}. \textit{J'ai l'habitude} de discuter avec mes amis ou d'écouter de la musique avec mes écouteurs. Les cours commencent à 8h45.\\
\lspbottomrule
\end{tabularx}
\caption{Sample NMT output for a short textbook excerpt} 
\label{tab:carre:4}
\end{table}

Errors include overly literal translations like \textit{une journée typique de ma vie}. The plural would work best in French and \textit{typique} needs rephrasing: \textit{mes journées de lycéen} (literally: ‘my days as a pupil’) would be a good solution. As you may note, literal translations are all the more inappropriate as the expression \textit{a typical day} is more or less fixed in the source language. For idioms like \textit{to give you an idea}, you will need to find an idiomatic expression in the target language: something like \textit{pour vous en donner un aperçu}.

Some of the errors are more linked to grammar and language use. Clitic pronouns like \textit{en} would be needed in the French text, and we could for instance improve \textit{donc j’arrive tôt} by turning it into \textit{donc j’y arrive en avance} (‘so I arrive in advance’). On the other hand, possessives are used in English even when possession is implicitly retrieved from the rest of the text, in which case the definite is preferred in French (\textit{je quitte \textbf{la} maison} ‘I leave the house’ rather than ‘my house’). Language use also concerns the lexicon, and while it is common in English to refer to manner of motion (\textit{a 15-minute walk to school}), French is usually more neutral (\textit{15 minutes de trajet pour l’école} ‘15 minutes of journey for the school’) with precisions added only if necessary (e.g. \textit{à pied} ‘on foot’).

\begin{sloppypar}
There are many more possible examples, but the above will hopefully be enough to show that although the French NMT output looks good enough, with no major grammatical or lexical mistakes, there is still a lot of room for improvement. Finding out what learners can and cannot correct will certainly be illuminating for teachers (see \citet{LoockLéchauguette2021} on this point).
\end{sloppypar}

We encourage readers to get NMT outputs for their own language pair, translating in the first instance into their L1. Activities including NMT outputs in an L2 in situated tasks can be used at a later stage of language learning, especially if the task involves detecting and fixing errors in the output.

\section{Conclusions}\label{sec:carre:7}

In this chapter we have presented some of the main findings of research to date into the use of MT in language learning, focusing on more recent sources that take into account the progress made in MT since the arrival on the scene of NMT. We have also offered some basic tips on using MT in language learning, before building on the pragmatic approach to quality evaluation presented in \textcitetv{chapters/rossi}, focusing on what it implies for second and foreign language learners. Unlike specialized translators, who may not be given a choice about the tools they use in current translation scenarios,  language learners have choices, but they first need to decide whether and when to use MT. To this end, we presented a list of situation-based parameters that can help them make this decision. We also contrasted MT with complementary online tools such as dictionaries and corpora, stressing the relative merits of each. Finally, we proposed activities to harness the potential of NMT and include it in the second or foreign language classroom.

\printbibliography[heading=subbibliography,notkeyword=this]

\end{document}
