\documentclass[output=paper]{langscibook}
\ChapterDOI{10.5281/zenodo.6759972}

\author{Dorothy Kenny\orcid{0000-0002-4793-9256}\affiliation{Dublin City University}}
\title{Introduction}
\abstract{In this Introduction I set out the rationale for this book and suggest ways in which readers might approach the material it contains.}


\begin{document}
\maketitle

\section{Why this book?}
Multilingualism is a foundational value of the European Union. In practice, it depends on multiple pillars, including language learning and translation. In recent years, both of these pillars have been profoundly affected by the continued development of machine translation. Machine translation, if used wisely, can serve to support language learning efforts. It can provide access to more texts, originally written in more languages, to more users than ever before. And it can help professional human translators to work more productively, given the right circumstances. There is a danger, however, that uncritical use of the technology could set back individual efforts to learn languages. It could also be used in ways that gloss over potential problems with the output it produces, so that consumers don't realize when texts they are reading contain machine-translation induced inaccuracies or biases. And if excessively hyped, it could discourage new blood from entering the translation profession. 

In order to harness the benefits of machine translation without jeopardizing other valuable pillars of multilingualism, we take the view, following \citet{BowkerCiro2019}, that good multilingual citizenship and professional translation must be underpinned by \textit{machine translation literacy}. Different levels of machine translation literacy will be appropriate for different people. \citet{BowkerCiro2019} are concerned with the international research community, for example. Here we target two main constituencies: more “occasional” users of machine translation --  those who use the technology for \textit{ad hoc} information gathering purposes, or even unwittingly when they consume machine translated text without realizing it, or casually in language learning contexts; and people who are either already working as translators or training to become translators. We take the view that all users of machine translation should have some basic understanding of why the technology is important, and where it fits into the maintenance of multilingual regimes. And all users should have some basic understanding of how the technology works, so they can use it intelligently and avoid common pitfalls. Some users, who may wish to engage more deeply with the technology, may benefit from knowing how to get the best out of machine translation, for example, by  writing texts in a way that makes them easier to translate by machine. The same users might also be interested in ways to improve machine translation outputs. Those working in, or about to join, the translation industry, will have a particular interest in evaluating machine translation output, in order to gauge whether it is “fit for purpose”. They might even get involved in integrating machine translation into the workflow of their company or need to know how to customize machine translation so that they can better serve the needs of particular clients. They will also be interested in how machine translation might impact on their working conditions. Such readers require more in-depth knowledge of the technology itself, and of the techniques and tools they can use to implement it. All users should have some basic knowledge of the ethical issues that arise when we use machine translation, for different reasons. Some users may be concerned about the possibility of cheating: in what cases might the use of machine translation constitute a breach of trust in educational environments, for example? Others, mainly professional translators, may have to consider how the use of certain types of machine translation might constitute a breach of contract. And everybody has to be concerned these days about protecting the privacy and data rights of others. Contemporary machine translation is also one of the many technologies that can be implicated in processes that degrade our natural environment. And it has been known to produce biased outputs, preferring male to female forms, for example. Like all communication technologies, it can be used for nefarious causes or positive humanitarian purposes. These are issues that concern us all.

\section{How to use this book}
In this book we attempt to guide readers through all of the above issues. We do not assume any prior knowledge of either translation in general or machine translation in particular. When we do move on to the more technical aspects of machine translation, and especially neural machine translation -- a state-of-the-art translation technology based on artificial neural networks -- we first describe these technologies without recourse to mathematical concepts that may not be familiar to readers. We attempt to use insightful explanations, and especially metaphors, that will help readers understand the general concepts that inform the area. In so doing, we provide a gentle introduction to the contemporary world of machine learning. We gradually build up a rich picture of machine translation. In general, readers will find the earlier chapters less specialized than the later ones, and earlier sections of later chapters less specialized than subsequent sections. Some readers may be able to skip some chapters completely, but even machine translation specialists might benefit from reading “non-technical” chapters, for example, on machine translation and ethics.

\section{The structure of this book}
This book opens with Olga Torres-Hostench's discussion in Chapter 1 of multilingualism, what it means, and how it is operationalized, in particular in the European Union. She makes the case for the considered integration of machine translation into both language learning and translation. In Chapter 2, Dorothy Kenny discusses translation in general and machine translation in particular, aiming to dispel some myths about translation before gently introducing the reader to the basic concepts behind contemporary approaches to machine translation, including artificial intelligence and machine learning. Chapter 3, by Caroline Rossi and Alice Carré, brings us into the world of machine translation evaluation, an area of considerable scientific and economic importance. In Chapter 4, Pilar Sánchez-Gijón and Dorothy Kenny address the ways in which we can make translation easier for machines before the fact, while in Chapter 5, Sharon O'Brien guides us through approaches to after-the-fact improvement of machine translated texts through the activity known as \textit{post-editing}. Chapter 6, by Joss Moorkens, discusses the many ethical issues that arise in contexts where machine translation is used. Chapter 7 is the most technical chapter in the book. In it, Juan Antonio Pérez-Ortiz, Mikel Forcada and Felipe Sánchez-Martínez explain how neural machine translation works, covering the basic techniques that are most commonly used in contemporary systems. Gema Ramírez-Sánchez's guide to custom neural machine translation follows in Chapter 8. The volume closes with a chapter dedicated to machine translation and language learning, written by Alice Carré, Dorothy Kenny, Caroline Rossi, Pilar Sánchez-Gijón and Olga Torres-Hostench.

\section*{Accompanying resources}
Each chapter of this book is accompanied by a set of interactive activities accessible through the MultiTraiNMT website at \url{http://www.multitrainmt.eu/}. A permanent link to these activities can be found at \url{https://ddd.uab.cat/record/257869}. Most of the activities can be completed on a self-access basis, although some will benefit from the guidance of a teacher.

A special pedagogical platform known as MutNMT has also been created as part of the MultiTraiNMT project. It is designed to help users learn how to train, customize and evaluate neural machine translation systems. It is accessible through the MultiTraiNMT website, and will be of particular significance to readers of Chapters 7 and 8 of this book.

\printbibliography[heading=subbibliography,notkeyword=this]

\end{document}
